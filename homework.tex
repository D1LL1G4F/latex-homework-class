\documentclass[11pt]{exam}
\newcommand{\myname}{Matej Kňazík} %Write your name in here
\newcommand{\myUCO}{506481} %write your UCO in here
\newcommand{\myhwtype}{Homework}
\newcommand{\myhwnum}{5} %Homework set number
\newcommand{\myclass}{IV054 2020}
\newcommand{\mylecture}{}
\newcommand{\mysection}{}

% Prefix for numedquestion's
\newcommand{\questiontype}{Question}

% Use this if your "written" questions are all under one section
% For example, if the homework handout has Section 5: Written Questions
% and all questions are 5.1, 5.2, 5.3, etc. set this to 5
% Use for 0 no prefix. Redefine as needed per-question.
\newcommand{\writtensection}{0}

\usepackage{amsmath, amsfonts, amsthm, amssymb}  % Some math symbols
\usepackage{enumerate}
\usepackage{enumitem}
\usepackage{graphicx}
\usepackage{hyperref}
\usepackage[all]{xy}
\usepackage{wrapfig}
\usepackage{fancyvrb}
\usepackage[T1]{fontenc}
\usepackage{listings}

\usepackage{centernot}
\usepackage{mathtools}
\DeclarePairedDelimiter{\ceil}{\lceil}{\rceil}
\DeclarePairedDelimiter{\floor}{\lfloor}{\rfloor}
\DeclarePairedDelimiter{\card}{\vert}{\vert}


\setlength{\parindent}{0pt}
\setlength{\parskip}{5pt plus 1pt}
\pagestyle{empty}

\def\indented#1{\list{}{}\item[]}
\let\indented=\endlist

\newcounter{questionCounter}
\newcounter{partCounter}[questionCounter]

\newenvironment{namedquestion}[1][\arabic{questionCounter}]{%
    \addtocounter{questionCounter}{1}%
    \setcounter{partCounter}{0}%
    \vspace{.2in}%
        \noindent{\bf #1}%
    \vspace{0.3em} \hrule \vspace{.1in}%
}{}

\newenvironment{numedquestion}[0]{%
	\stepcounter{questionCounter}%
    \vspace{.2in}%
        \ifx\writtensection\undefined
        \noindent{\bf \questiontype \; \arabic{questionCounter}. }%
        \else
          \if\writtensection0
          \noindent{\bf \questiontype \; \arabic{questionCounter}. }%
          \else
          \noindent{\bf \questiontype \; \writtensection.\arabic{questionCounter} }%
        \fi
    \vspace{0.3em} \hrule \vspace{.1in}%
}{}

\newenvironment{alphaparts}[0]{%
  \begin{enumerate}[label=\textbf{(\alph*)}]
}{\end{enumerate}}

\newenvironment{arabicparts}[0]{%
  \begin{enumerate}[label=\textbf{\arabic{questionCounter}.\arabic*})]
}{\end{enumerate}}

\newenvironment{questionpart}[0]{%
  \item
}{}

\newcommand{\answerbox}[1]{
\begin{framed}
\vspace{#1}
\end{framed}}

\pagestyle{head}

\headrule
\header{\textbf{\myclass\ \mylecture\mysection}}%
{\textbf{\myname\ (\myUCO)}}%
{\textbf{\myhwtype\ \myhwnum}}

\begin{document}
\thispagestyle{plain}
\begin{center}
  {\Large \myclass{} \myhwtype{} \myhwnum} \\
  \myname{} (\myUCO{}) \\
  \today
\end{center}


%Here you can enter answers to homework questions

\begin{numedquestion}
  \begin{alphaparts}
    \item
    $w = 506481; e = 11; n = 1147$ \\
    $w_0=506; w_1=481$ \\
    $c_0= w_0^e$ mod $n$ \\
    $c_1= w_1^e$ mod $n$  \\
    $c_0= 670$ \\
    $c_1= 481$ \\
    $c =$ concat $(c_0,c_1)$ \\
    \textbf{encrypted: $c=135675$} \\
    $d = e^{-1} mod (p-1)(q-1)$  \\
    $d = 11^{-1}$ mod  $30 \times 36$ \\
    $d = 11^{-1}$ mod $1080$ \\
    $d = 491$ \\
    $w_0=c_0^d$ mod $ n$ \\
    $w_0=506=670^491$ mod $ 1147$ \\
    $w_1=c_1^d$ mod $ n$ \\
    $w_1=481=481^491$ mod $ 1147$ \\
    $w=$ concat $(w_0,w_1)$ \\
    \textbf{decrypted: $w = 506481$}
    
    \item
    
  \end{alphaparts}
\end{numedquestion}

\begin{numedquestion}
    $65201327 = p \times q; \\
    \phi(65201327)=65185176=(p-1)\times(q-1) \\
    \rightarrow 65201327 = 7933 \times 8219$ \\
    \\
    $134635439 = p \times q; \\
    \phi(134635439)=134610840=(p-1)\times(q-1) \\
    \rightarrow  134635439 = 8219 \times 16381$ \\
    \\
    $122176133 = p \times q; \\
    \phi(122176133)=122152800=(p-1)\times(q-1) \\
    \rightarrow 122176133 = 7933 \times 15401$ \\
    \\
    $122237737 = p \times q; \\
    \phi(122237737)=122214400=(p-1)\times(q-1) \\
    \rightarrow 122237737 = 7937 \times 15401$ \\
    \\
    $99633161 = p \times q; \\
    \phi(99633161)=99612672=(p-1)\times(q-1) \\
    \rightarrow 99633161 = 7937 \times 12553$ \\ 
\end{numedquestion}

%You are given n = 633917, for which you know that it is a product of two primes and
%phi(n) = 632256. Find factors of n without using brute force. Explain your calculations.

\begin{numedquestion}
  $n = 633917 = p \times q$ \\
  $phi(n) = 632256 = (p-1)\times(q-1)$ \\
  $\rightarrow 633917 = 593 \times 1069$ \\ 
\end{numedquestion}


% if you do not solve some of the questions use this command to increment counter
% \setcounter{questionCounter}{4}
% \begin{numedquestion}
%   Questions 2 and 3 were not solved, this is an answer to question 5.
% \end{numedquestion}

\setcounter{questionCounter}{5}
\begin{numedquestion}
key: \textbf{tabularecta} \\
key length: \textbf{11} \\
Decipher: \\
BABBA GESSU CCESS FULCR YPTAN ALYSI SOFTH EVIGE NEREC IPHER WASPR OBABL YACHI EVEDI
NEIGH TEENF IFTYF OURSO ONAFT ERHIS SPATW ITHTH WAITE SBUTH ISDIS COVER YWENT COMPL
ETELY UNREC OGNIZ EDBEC AUSEH ENEVE RPUBL ISHED ITTHE DISCO VERYC AMETO LIGHT ONLYI
NTHET WENTI ETHCE NTURY WHENS CHOLA RSEXA MINED BABBA GESEX TENSI VENOT ESINT HEMEA
NTIME HISTE CHNIQ UEWAS INDEP ENDEN TLYDI SCOVE REDBY FRIED RICHW ILHEL MKASI SKIAR
ETIRE DOFFI CERIN THEPR USSIA NARMY EVERS INCEW HENHE PUBLI SHEDH ISCRY PTANA LYTIC
BREAK THROU GHIND IEGEH EIMSC HRIFT ENUND DIEDE CHIFF RIRKU NSTSE CRETW RITIN GANDT
HEART OFDEC IPHER INGTH ETECH NIQUE HASBE ENKNO WNAST HEKAS ISKIT ESTAN DBABB AGESC
ONTRI BUTIO NHASB EENLA RGELY IGNOR EDAND WHYDI DBABB AGEFA ILTOP UBLIC IZEHI SCRAC
KINGO FSUCH AVITA LCIPH ERHEC ERTAI NLYHA DAHAB ITOFN OTFIN ISHIN GPROJ ECTSA NDNOT
PUBLI SHING HISDI SCOVE RIESW HICHM IGHTS UGGES TTHAT THISI SJUST ONEMO REEXA MPLEO
FHISL ACKAD AISIC ALATT ITUDE HOWEV ERTHE REISA NALTE RNATI VEEXP LANAT IONHI SDISC
OVERY OCCUR REDSO ONAFT ERTHE OUTBR EAKOF THECR IMEAN WARAN DONET HEORY ISTHA TITGA
VETHE BRITI SHACL EARAD VANTA GEOVE RTHEI RRUSS IANEN EMYIT ISQUI TEPOS SIBLE THATB
RITIS HINTE LLIGE NCEDE MANDE DTHAT BABBA GEKEE PHISW ORKSE CRETT HUSPR OVIDI NGTHE
MWITH ANINE YEARH EADST ARTOV ERTHE RESTO FTHEW ORLDI FTHIS WASTH ECASE THENI TWOUL
DFITI NWITH THELO NGSTA NDING TRADI TIONO FHUSH INGUP CODEB REAKI NGACH IEVEM ENTSI
NTHEI NTERE STSOF NATIO NALSE CURIT YAPRA CTICE THATH ASCON TINUE DINTO THETW ENTIE
THCEN TURYS IMONS INGHT HECOD EBOOK

Corrected spaces clearly shows english text: \\
babb ages successful crypt analysis of the vigenere cipher was probably achieved in eighteen fifty four soon after his spat with th wait es but his discovery went completely unrecognized because he never published it the discovery came to light only in the twentieth century when scholars examined babb ages extensive notes in the meantime his technique was independently discovered by fried rich wilhelm kasiskia retired officer in the prussian army ever since when he published his crypt an alytic tcaa aaa

\end{numedquestion}



\end{document}