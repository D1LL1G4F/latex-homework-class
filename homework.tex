\documentclass[11pt]{exam}
\newcommand{\myname}{Matej Kňazík} %Write your name in here
\newcommand{\myUCO}{506481} %write your UCO in here
\newcommand{\myhwtype}{Homework}
\newcommand{\myhwnum}{3} %Homework set number
\newcommand{\myclass}{IV054 2020}
\newcommand{\mylecture}{}
\newcommand{\mysection}{}

% Prefix for numedquestion's
\newcommand{\questiontype}{Question}

% Use this if your "written" questions are all under one section
% For example, if the homework handout has Section 5: Written Questions
% and all questions are 5.1, 5.2, 5.3, etc. set this to 5
% Use for 0 no prefix. Redefine as needed per-question.
\newcommand{\writtensection}{0}

\usepackage{amsmath, amsfonts, amsthm, amssymb}  % Some math symbols
\usepackage{enumerate}
\usepackage{enumitem}
\usepackage{graphicx}
\usepackage{hyperref}
\usepackage[all]{xy}
\usepackage{wrapfig}
\usepackage{fancyvrb}
\usepackage[T1]{fontenc}
\usepackage{listings}

\usepackage{centernot}
\usepackage{mathtools}
\DeclarePairedDelimiter{\ceil}{\lceil}{\rceil}
\DeclarePairedDelimiter{\floor}{\lfloor}{\rfloor}
\DeclarePairedDelimiter{\card}{\vert}{\vert}


\setlength{\parindent}{0pt}
\setlength{\parskip}{5pt plus 1pt}
\pagestyle{empty}

\def\indented#1{\list{}{}\item[]}
\let\indented=\endlist

\newcounter{questionCounter}
\newcounter{partCounter}[questionCounter]

\newenvironment{namedquestion}[1][\arabic{questionCounter}]{%
    \addtocounter{questionCounter}{1}%
    \setcounter{partCounter}{0}%
    \vspace{.2in}%
        \noindent{\bf #1}%
    \vspace{0.3em} \hrule \vspace{.1in}%
}{}

\newenvironment{numedquestion}[0]{%
	\stepcounter{questionCounter}%
    \vspace{.2in}%
        \ifx\writtensection\undefined
        \noindent{\bf \questiontype \; \arabic{questionCounter}. }%
        \else
          \if\writtensection0
          \noindent{\bf \questiontype \; \arabic{questionCounter}. }%
          \else
          \noindent{\bf \questiontype \; \writtensection.\arabic{questionCounter} }%
        \fi
    \vspace{0.3em} \hrule \vspace{.1in}%
}{}

\newenvironment{alphaparts}[0]{%
  \begin{enumerate}[label=\textbf{(\alph*)}]
}{\end{enumerate}}

\newenvironment{arabicparts}[0]{%
  \begin{enumerate}[label=\textbf{\arabic{questionCounter}.\arabic*})]
}{\end{enumerate}}

\newenvironment{questionpart}[0]{%
  \item
}{}

\newcommand{\answerbox}[1]{
\begin{framed}
\vspace{#1}
\end{framed}}

\pagestyle{head}

\headrule
\header{\textbf{\myclass\ \mylecture\mysection}}%
{\textbf{\myname\ (\myUCO)}}%
{\textbf{\myhwtype\ \myhwnum}}

\begin{document}
\thispagestyle{plain}
\begin{center}
  {\Large \myclass{} \myhwtype{} \myhwnum} \\
  \myname{} (\myUCO{}) \\
  \today
\end{center}


%Here you can enter answers to homework questions

\begin{numedquestion}
  \begin{alphaparts}
    \item $n = 7;$ deg($g(x)$) = k = 4; $g_0=1;g_1=0;g_2=1;g_3=1;g_4=1$
    \[
    G =
        \begin{bmatrix}
        1 & 0 & 1 & 1 & 1 & 0 & 0\\
        0 & 1 & 0 & 1 & 1 & 1 & 0\\
        0 & 0 & 1 & 0 & 1 & 1 & 1\\
        \end{bmatrix}
    \]
    \item $h(x)=1+x^2+x^3$
    \[
     H =
        \begin{bmatrix}
        1 & 1 & 0 & 1 & 0 & 0 & 0\\
        0 & 1 & 1 & 0 & 1 & 0 & 0\\
        0 & 0 & 1 & 1 & 0 & 1 & 0\\
        0 & 0 & 0 & 1 & 1 & 0 & 1\\
        \end{bmatrix}
    \]
    \item $c=1011100$
  \end{alphaparts}
\end{numedquestion}

\begin{numedquestion}
  \begin{alphaparts}
    \item 
    $G$ is a $k \times n $ matrix \\
    $k = 2$ \\
    $n = 5$ \\
    $d = 2$ \\
    \item 
        \begin{tabular}{ |c|c|c|c| } 
            \hline
            00000 & 01010 & 10101 & 11111 \\ 
            \hline
            00001 & 01011 & 10100 & \textbf{11110} \\ 
            \hline
            00010 & 01000 & 10111 & 11101 \\ 
            \hline
            00100 & 01110 & 10001 & 11011 \\ 
            \hline
            01000 & 00010 & 11101 & 10111 \\ 
            \hline
            10000 & 11010 & 00101 & 01111 \\ 
            \hline
            11000 & 10010 & 01101 & 00111 \\ 
            \hline
            00011 & 01001 & 10110 & 11100 \\ 
            \hline
        \end{tabular}
    \item
    coset leader: 00001 \\
    decoded: 11111
  \end{alphaparts}
\end{numedquestion}


% if you do not solve some of the questions use this command to increment counter
% \setcounter{questionCounter}{4}
% \begin{numedquestion}
%   Questions 2 and 3 were not solved, this is an answer to question 5.
% \end{numedquestion}


% if questions have subparts, use this command
\begin{numedquestion}
  \begin{alphaparts}
    \item Yes.
    \item No.
    \item Yes for binary linear codes.
  \end{alphaparts}
\end{numedquestion}

\begin{numedquestion}
\begin{alphaparts}
  \item 
  \[
  H =
  \begin{bmatrix}
1 & 1 & 0 & 1 & 0 & 0\\
1 & 0 & 1 & 0 & 1 & 0\\
1 & 1 & 1 & 0 & 0 & 1
\end{bmatrix}
  \]
  
  \item
  
  \[
  H^T =
  \begin{bmatrix}
1 & 1 & 1 \\
1 & 0 & 1 \\
0 & 1 & 1 \\
1 & 0 & 0 \\
0 & 1 & 0 \\
0 & 0 & 1 
\end{bmatrix}
  \]
  $e = 000010 $\\
  $e * H^T = 010 $
\end{alphaparts}
\end{numedquestion}

\begin{numedquestion}
  \begin{alphaparts}
    \item 
    \[
  C_1 =
      \begin{bmatrix}
    1 & 0 & 0 & 1 \\
    0 & 1 & 1 & 1 \\
    1 & 1 & 1 & 0 \\
    0 & 0 & 0 & 0 
    \end{bmatrix}
      \]
  \end{alphaparts}
\end{numedquestion}



\begin{numedquestion}

Since $C \subset C^\perp$ than the code is weakly self-dual code.
All codewords in such code have even weight.


\end{numedquestion}



\end{document}