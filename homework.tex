\documentclass[11pt]{exam}
\newcommand{\myname}{Matej Kňazík} %Write your name in here
\newcommand{\myUCO}{506481} %write your UCO in here
\newcommand{\myhwtype}{Homework}
\newcommand{\myhwnum}{9} %Homework set number
\newcommand{\myclass}{IV054 2020}
\newcommand{\mylecture}{} 
\newcommand{\mysection}{}

% Prefix for numedquestion's
\newcommand{\questiontype}{Question}

% Use this if your "written" questions are all under one section
% For example, if the homework handout has Section 5: Written Questions
% and all questions are 5.1, 5.2, 5.3, etc. set this to 5
% Use for 0 no prefix. Redefine as needed per-question.
\newcommand{\writtensection}{0}

\usepackage{amsmath, amsfonts, amsthm, amssymb}  % Some math symbols
\usepackage{enumerate}
\usepackage{enumitem}
\usepackage{graphicx}
\usepackage{hyperref}
\usepackage[all]{xy}
\usepackage{wrapfig}
\usepackage{fancyvrb}
\usepackage[T1]{fontenc}
\usepackage{listings}

\usepackage{centernot}
\usepackage{mathtools}
\DeclarePairedDelimiter{\ceil}{\lceil}{\rceil}
\DeclarePairedDelimiter{\floor}{\lfloor}{\rfloor}
\DeclarePairedDelimiter{\card}{\vert}{\vert}


\setlength{\parindent}{0pt}
\setlength{\parskip}{5pt plus 1pt}
\pagestyle{empty}

\def\indented#1{\list{}{}\item[]}
\let\indented=\endlist

\newcounter{questionCounter}
\newcounter{partCounter}[questionCounter]

\newenvironment{namedquestion}[1][\arabic{questionCounter}]{%
    \addtocounter{questionCounter}{1}%
    \setcounter{partCounter}{0}%
    \vspace{.2in}%
        \noindent{\bf #1}%
    \vspace{0.3em} \hrule \vspace{.1in}%
}{}

\newenvironment{numedquestion}[0]{%
	\stepcounter{questionCounter}%
    \vspace{.2in}%
        \ifx\writtensection\undefined
        \noindent{\bf \questiontype \; \arabic{questionCounter}. }%
        \else
          \if\writtensection0
          \noindent{\bf \questiontype \; \arabic{questionCounter}. }%
          \else
          \noindent{\bf \questiontype \; \writtensection.\arabic{questionCounter} }%
        \fi
    \vspace{0.3em} \hrule \vspace{.1in}%
}{}

\newenvironment{alphaparts}[0]{%
  \begin{enumerate}[label=\textbf{(\alph*)}]
}{\end{enumerate}}

\newenvironment{arabicparts}[0]{%
  \begin{enumerate}[label=\textbf{\arabic{questionCounter}.\arabic*})]
}{\end{enumerate}}

\newenvironment{questionpart}[0]{%
  \item
}{}

\newcommand{\answerbox}[1]{
\begin{framed}
\vspace{#1}
\end{framed}}

\pagestyle{head}

\headrule
\header{\textbf{\myclass\ \mylecture\mysection}}%
{\textbf{\myname\ (\myUCO)}}%
{\textbf{\myhwtype\ \myhwnum}}

\begin{document}
\thispagestyle{plain}
\begin{center}
  {\Large \myclass{} \myhwtype{} \myhwnum} \\
  \myname{} (\myUCO{}) \\
  \today
\end{center}


%Here you can enter answers to homework questions

\begin{numedquestion}
\begin{alphaparts}
    \item
    $(n,t) = (5,3)$ \\
    $p = 567997$ \\
    $\{x_i = i\}_{i=1}^5$ \\
    $a_1 = 3^{506481} \mod 101021$ \\
    $a_2 = 5^{506481} \mod 101021$ \\
    $a_1 = 53814$ \\
    $a_2 = 41819$ \\
    $S = 506481$ \\
    \[y_i = S + \sum_{j=1}^{t-1} a_j x_{i}^{j} \pmod{p}\]
    $y_i = a_2x_i^2 + a_1x_i + S \pmod{p}$ \\
    $y_1 = 41819 + 53814 + 506481 \pmod{567997}$ \\
    $y_2 = 41819 * 4 + 53814 * 2 + 506481 \pmod{567997}$ \\
    $y_3 = 41819 * 9 + 53814 * 3 + 506481 \pmod{567997}$ \\
    $y_4 = 41819 * 16 + 53814 * 4 + 506481 \pmod{567997}$ \\
    $y_5 = 41819 * 25 + 53814 * 5 + 506481 \pmod{567997}$ \\
    $y_1 = 34117$ \\
    $y_2 = 213388$ \\
    $y_3 = 476297$ \\
    $y_4 = 254847$ \\
    $y_5 = 117035$ \\
    $R_i = (x_i,y_i)$ \\
    \textbf{shares of the threshold scheme:} $(1,34117),(2,213388),(3,476297),(4,254847),(5,117035)$ \\
    \item
    $p = 567997$ \\
    $(1, 438827),(2, 273042),(3, 133864)$ \\
    $0 \equiv a_2 + a_1 + S - 438827 \pmod{567997}$ \\
    $0 \equiv 4a_2 + 2a_1 + S - 273042 \pmod{567997}$ \\
    $0 \equiv 9a_2 + 3a_1 + S - 133864 \pmod{567997}$ \\
    using CRT I get: \\
    $a_1 = 78303$ \\
    $a_2 = 297302$ \\
    \textbf{secret: }$S = 63222$
\end{alphaparts}
\end{numedquestion}


\setcounter{questionCounter}{2}
\begin{numedquestion}
\textbf{Yes}, it is possible. Example could be instance of threshold secret sharing scheme (14000,210) \\
$1/1 F \lor 3/10G \lor (2/10G \land 5/50C) \lor (1/10G \land 7/50C \land 15/100M)$ \\
\\
$(2/10G \land 5/50C) : (10,2) \land (50,5) \implies (2,2) $ \\
$(1/10G \land 7/50C \land 15/100M) : (10,1) \land (50,7) \land (100,15) \implies (3,3)$ \\
$1/1 F \lor 3/10G \lor (2/10G \land 5/50C) \lor (1/10G \land 7/50C \land 15/100M) : (1,1) \lor (10,3) \lor (2,2) \lor (3,3) \implies (4,1)$
\\
this means that we need to find such threshold secret sharing scheme $(n_s,t_s)$ that would substitute all of these: \\
$(10,2),(50,5),(2,2),(10,1),(50,7),(100,15),(3,3),(1,1),(10,3),(2,2),(3,3),(4,1)$ \\
we can create such threshold secret sharing scheme with: (100 * $t_s$/15,LCM(2,5,1,7,15,3)) \\
e.g to convert $2/10G : (10,2)$ to $(14000,210)$ each of the 10 generals would get 105 shares from (14000,210) etc... \\
since $t_s$ Least Common Multiple of all $t_i$ this can be done for all of the listed $(n_i,t_i)$




\end{numedquestion}

\begin{numedquestion}
I used Point addition to recursively determine all possible points. \\
$P + Q = R$ \\
$(x_p,y_p) + (x_q,y_q) = (x_r,y_r)$ \\
$\lambda = \frac{y_q-y_p}{x_q - x_p}$ \\
$x_r = \lambda^2 - x_p - x_q$ \\
$y_r = \lambda \times (x_p - x_r ) - y_p$ \\
Number of points of elliptic curve $E: y^2=x_3+3x+7 \pmod{113}$ is \textbf{126} (including the point at infinity).
\end{numedquestion}


% if you do not solve some of the questions use this command to increment counter
% \setcounter{questionCounter}{4}
% \begin{numedquestion}
%   Questions 2 and 3 were not solved, this is an answer to question 5.
% \end{numedquestion}



\end{document}