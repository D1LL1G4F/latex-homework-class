\documentclass[11pt]{exam}
\newcommand{\myname}{Matej Kňazík} %Write your name in here
\newcommand{\myUCO}{506481} %write your UCO in here
\newcommand{\myhwtype}{Homework}
\newcommand{\myhwnum}{6} %Homework set number
\newcommand{\myclass}{IV054 2020}
\newcommand{\mylecture}{} 
\newcommand{\mysection}{}

% Prefix for numedquestion's
\newcommand{\questiontype}{Question}

% Use this if your "written" questions are all under one section
% For example, if the homework handout has Section 5: Written Questions
% and all questions are 5.1, 5.2, 5.3, etc. set this to 5
% Use for 0 no prefix. Redefine as needed per-question.
\newcommand{\writtensection}{0}

\usepackage{amsmath, amsfonts, amsthm, amssymb}  % Some math symbols
\usepackage{enumerate}
\usepackage{enumitem}
\usepackage{graphicx}
\usepackage{hyperref}
\usepackage[all]{xy}
\usepackage{wrapfig}
\usepackage{fancyvrb}
\usepackage[T1]{fontenc}
\usepackage{listings}

\usepackage{centernot}
\usepackage{mathtools}
\DeclarePairedDelimiter{\ceil}{\lceil}{\rceil}
\DeclarePairedDelimiter{\floor}{\lfloor}{\rfloor}
\DeclarePairedDelimiter{\card}{\vert}{\vert}


\setlength{\parindent}{0pt}
\setlength{\parskip}{5pt plus 1pt}
\pagestyle{empty}

\def\indented#1{\list{}{}\item[]}
\let\indented=\endlist

\newcounter{questionCounter}
\newcounter{partCounter}[questionCounter]

\newenvironment{namedquestion}[1][\arabic{questionCounter}]{%
    \addtocounter{questionCounter}{1}%
    \setcounter{partCounter}{0}%
    \vspace{.2in}%
        \noindent{\bf #1}%
    \vspace{0.3em} \hrule \vspace{.1in}%
}{}

\newenvironment{numedquestion}[0]{%
	\stepcounter{questionCounter}%
    \vspace{.2in}%
        \ifx\writtensection\undefined
        \noindent{\bf \questiontype \; \arabic{questionCounter}. }%
        \else
          \if\writtensection0
          \noindent{\bf \questiontype \; \arabic{questionCounter}. }%
          \else
          \noindent{\bf \questiontype \; \writtensection.\arabic{questionCounter} }%
        \fi
    \vspace{0.3em} \hrule \vspace{.1in}%
}{}

\newenvironment{alphaparts}[0]{%
  \begin{enumerate}[label=\textbf{(\alph*)}]
}{\end{enumerate}}

\newenvironment{arabicparts}[0]{%
  \begin{enumerate}[label=\textbf{\arabic{questionCounter}.\arabic*})]
}{\end{enumerate}}

\newenvironment{questionpart}[0]{%
  \item
}{}

\newcommand{\answerbox}[1]{
\begin{framed}
\vspace{#1}
\end{framed}}

\pagestyle{head}

\headrule
\header{\textbf{\myclass\ \mylecture\mysection}}%
{\textbf{\myname\ (\myUCO)}}%
{\textbf{\myhwtype\ \myhwnum}}

\begin{document}
\thispagestyle{plain}
\begin{center}
  {\Large \myclass{} \myhwtype{} \myhwnum} \\
  \myname{} (\myUCO{}) \\
  \today
\end{center}


%Here you can enter answers to homework questions

\begin{numedquestion}

    $x \equiv 6 \pmod{17}$ \\
	$x \equiv 3 \pmod{7}$ \\
	$x \equiv 9 \pmod{11}$ \\
	\\
	$x \equiv b_1 \pmod{n_1}$ \\
	$x \equiv b_2 \pmod{n_2}$ \\
	$x \equiv b_3 \pmod{n_3}$ \\
	\\
	$N = n_1 \times n_2 \times n_3 = 17 \times 7 \times 11 = 1309$
	\begin{table}[h!]
    \begin{center}
    \label{tab:table1}
    \begin{tabular}{|c|c|c|c|c|}
      i & $b_i$ & $N_i = \frac{N}{n_i}$ & $N_i' = $ Inverse of $N_i$ & $b_i \times N_i \times N_i'$\\
      \hline
      1 & 6 & 77 & 2 & 924\\
      2 & 3 & 187 & 3 & 1683\\
      3 & 9 & 119 & 5 & 5355\\
    \end{tabular}
    \end{center}
    \end{table}
    
$b = b_1 \times N_1 \times N_1' + b_2 \times N_2 \times N_2' + b_3 \times N_3 \times N_3'$ \\
$b = 924 + 1683 + 5355$ \\
$b = 7962$ \\
$x \equiv b \pmod{N}$ \\
$x \equiv 7962 \pmod{1309}$ \\
$x \equiv 108 \pmod{1309}$ \\
\\
$108 \equiv 6 \pmod{17}$ \\
$108 \equiv 3 \pmod{7}$ \\
$108 \equiv 9 \pmod{11}$ \\
\\
$x = 108$


	
\end{numedquestion}

\begin{numedquestion}
$y$ is a quadratic residue modulo some odd prime $p$, if there exists $x$ such that 
$x^2 \equiv y \pmod{p}$
\\
The Legendre symbol $(\frac{y}{p})$ is defined to be 0 if $y$ is a multiple of $p$, 1 if $y$ is a quadratic residue modulo $p$, and $-1$ otherwise.
\\
\begin{math}
\left(\frac{1}{8009}\right) = 1 \\
1^2 \equiv 1 \pmod{8009} \\
\left(\frac{2}{8009}\right) = 1 \\
3820^2 \equiv 2 \pmod{8009} \\
\left(\frac{3}{8009}\right) = -1 \\
3 \textit{ is a quadratic non-residue } \pmod{8009} \\
\left(\frac{4}{8009}\right) = 1 \\
2^2 \equiv 4 \pmod{8009} \\
\left(\frac{5}{8009}\right) = 1 \\
179^2 \equiv 5 \pmod{8009} \\
\left(\frac{6}{8009}\right) = -1 \\
6 \textit{ is a quadratic non-residue } \pmod{8009} \\
\left(\frac{7}{8009}\right) = 1 \\
2509^2 \equiv 7 \pmod{8009} \\
\left(\frac{8}{8009}\right) = 1 \\
369^2 \equiv 8 \pmod{8009} \\
\left(\frac{9}{8009}\right) = 1 \\
3^2 \equiv 9 \pmod{8009} \\
\left(\frac{10}{8009}\right) = 1 \\
30152 \equiv 10 \pmod{8009} \\
\end{math}
\textbf{Quadratic residuis module the prime 8009 are \{1,2,4,5,7,8,9,10\}}
\end{numedquestion}

\begin{numedquestion}
  \begin{alphaparts}
  \item
  $n = 698069$ \\
  $m = 506481$ \\
  $p \times q = 887 \times 787$ \\
  $c = m^2 \pmod{n}$ \\
  $c = 506481^2 \pmod{698069}$ \\
  $c = 256523003361 \pmod{698069}$ \\
  \textbf{cyphertext: }$c = 97586$ \\
  $m_p = \sqrt{c} \pmod{p} = \pm c^{\frac{p+1}{4}} \pmod{p}$ \\
  $m_q = \sqrt{c} \pmod{q} = \pm c^{\frac{q+1}{4}} \pmod{q}$ \\
  $m_p = \sqrt{97586} \pmod{887} = \pm 97586^{\frac{888}{4}} \pmod{887}$ \\
  $m_q = \sqrt{97586} \pmod{787} = \pm 97586^{\frac{788}{4}} \pmod{787}$ \\
  $m_p = \pm 4$ \\
  $m_q = \pm 440$ \\
  \\
  $m_1p = 4$ \\
  $m_1q = 440$ \\
  \\
  $m_2p = -4$ \\
  $m_2q = 440$ \\
  \\
  $m_3p = 4$ \\
  $m_3q = -440$ \\
  \\
  $m_4p = -4$ \\
  $m_4q = -440$ \\
  \\
  extended Euclidean algorithm = EEA \\
  EEA(p,q) = \{$y_p,y_q$\} \\
  EEA(887, 787) = \{-181, 204\} \\
  $m_1=y_p \times p \times m_1q + y_q \times q \times m_1p \pmod{n}$ \\
  $m_2=y_p \times p \times m_2q + y_q \times q \times m_2p \pmod{n}$ \\
  $m_3=y_p \times p \times m_3q + y_q \times q \times m_3p \pmod{n}$ \\
  $m_4=y_p \times p \times m_4q + y_q \times q \times m_4p \pmod{n}$ \\
  \\
  $m_1=-181 \times 887 \times 440 + 204 \times 787 \times 4 \pmod{698069}$ \\
  $m_2=-181 \times 887 \times 440 + 204 \times 787 \times -4 \pmod{698069}$ \\
  $m_3=-181 \times 887 \times -440 + 204 \times 787 \times 4 \pmod{698069}$ \\
  $m_4=-181 \times 887 \times -440 + 204 \times 787 \times -4 \pmod{698069}$ \\
  \\
  $m_1= -69998488 \pmod{698069} = 506481$ \\
  $m_2= -71282872 \pmod{698069} = 618235$\\
  $m_3= 71282872 \pmod{698069} = 79834$ \\
  $m_4= 69998488 \pmod{698069} = 191588$ \\
  \textbf{original plaintext:} $m_1 = m = 506481$
  \item
  $m = 506481$ \\
  $p = 567899$ \\
  $q = 2$ \\
  $x = 12345$ \\
  $r = 938$ \\
  $y = q^x \pmod{p}$ \\
  $y = 2^{12345} \pmod{567899}$ \\
  $y = 222588$ \\
  $a = q^r \pmod{p}$ \\
  $a = 2^{938} \pmod{567899}$ \\
  $a = 201104$ \\
  $b = y^r \times m \pmod{p}$ \\
  $b = 222588^{938} \times 506481 \pmod{567899}$ \\
  $b = 566491$ \\
  \textbf{ciphertext:} $c = (a,b) = (201104,566491)$
  
  \end{alphaparts}
\end{numedquestion}

\setcounter{questionCounter}{4}
\begin{numedquestion}
  \begin{alphaparts}
    \item
     P(at least two) = 1 - P(everybody has different) \\
     P(everybody has different) = $\frac{7}{7} \times \frac{6}{7} \times \frac{5}{7} \times \frac{4}{7} \times \frac{3}{7}$ \\
     P(everybody has different) $\approx 0.149937526$ \\
     P(at least two) $\approx 1 - 0.149937526$ \\
     P(at least two) $\approx 0.850062473969$
    \item
    P(exactly two) = $\frac{7}{7} \times \frac{1}{7} \times \frac{6}{7} \times \frac{5}{7} \times \frac{4}{7}$ \\
    P(exactly two) = $120/2401$ \\
    P(exactly two) $\approx 0.0499791753436$ \\
    \item
    P(at least three) = P(at least two) - P(exactly two) \\
    P(at least three) $\approx 0.8$
  \end{alphaparts}
\end{numedquestion}


% if you do not solve some of the questions use this command to increment counter
% \setcounter{questionCounter}{4}
% \begin{numedquestion}
%   Questions 2 and 3 were not solved, this is an answer to question 5.
% \end{numedquestion}



\end{document}