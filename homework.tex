\documentclass[11pt]{exam}
\newcommand{\myname}{Matej Kňazík} %Write your name in here
\newcommand{\myUCO}{506481} %write your UCO in here
\newcommand{\myhwtype}{Homework}
\newcommand{\myhwnum}{1} %Homework set number
\newcommand{\myclass}{IV054 2020}
\newcommand{\mylecture}{}
\newcommand{\mysection}{}

% Prefix for numedquestion's
\newcommand{\questiontype}{Question}

% Use this if your "written" questions are all under one section
% For example, if the homework handout has Section 5: Written Questions
% and all questions are 5.1, 5.2, 5.3, etc. set this to 5
% Use for 0 no prefix. Redefine as needed per-question.
\newcommand{\writtensection}{0}

\usepackage{amsmath, amsfonts, amsthm, amssymb}  % Some math symbols
\usepackage{enumerate}
\usepackage{enumitem}
\usepackage{graphicx}
\usepackage{hyperref}
\usepackage[all]{xy}
\usepackage{wrapfig}
\usepackage{fancyvrb}
\usepackage[T1]{fontenc}
\usepackage{listings}

\usepackage{centernot}
\usepackage{mathtools}
\DeclarePairedDelimiter{\ceil}{\lceil}{\rceil}
\DeclarePairedDelimiter{\floor}{\lfloor}{\rfloor}
\DeclarePairedDelimiter{\card}{\vert}{\vert}


\setlength{\parindent}{0pt}
\setlength{\parskip}{5pt plus 1pt}
\pagestyle{empty}

\def\indented#1{\list{}{}\item[]}
\let\indented=\endlist

\newcounter{questionCounter}
\newcounter{partCounter}[questionCounter]

\newenvironment{namedquestion}[1][\arabic{questionCounter}]{%
    \addtocounter{questionCounter}{1}%
    \setcounter{partCounter}{0}%
    \vspace{.2in}%
        \noindent{\bf #1}%
    \vspace{0.3em} \hrule \vspace{.1in}%
}{}

\newenvironment{numedquestion}[0]{%
	\stepcounter{questionCounter}%
    \vspace{.2in}%
        \ifx\writtensection\undefined
        \noindent{\bf \questiontype \; \arabic{questionCounter}. }%
        \else
          \if\writtensection0
          \noindent{\bf \questiontype \; \arabic{questionCounter}. }%
          \else
          \noindent{\bf \questiontype \; \writtensection.\arabic{questionCounter} }%
        \fi
    \vspace{0.3em} \hrule \vspace{.1in}%
}{}

\newenvironment{alphaparts}[0]{%
  \begin{enumerate}[label=\textbf{(\alph*)}]
}{\end{enumerate}}

\newenvironment{arabicparts}[0]{%
  \begin{enumerate}[label=\textbf{\arabic{questionCounter}.\arabic*})]
}{\end{enumerate}}

\newenvironment{questionpart}[0]{%
  \item
}{}

\newcommand{\answerbox}[1]{
\begin{framed}
\vspace{#1}
\end{framed}}

\pagestyle{head}

\headrule
\header{\textbf{\myclass\ \mylecture\mysection}}%
{\textbf{\myname\ (\myUCO)}}%
{\textbf{\myhwtype\ \myhwnum}}

\begin{document}
\thispagestyle{plain}
\begin{center}
  {\Large \myclass{} \myhwtype{} \myhwnum} \\
  \myname{} (\myUCO{}) \\
  \today
\end{center}


%Here you can enter answers to homework questions

\begin{numedquestion}
  \begin{alphaparts}
    \item \textbf{No.} But if we would add codeword 000 we would make it linear.
    \item \textbf{No.} For example if we invert linear code $C = {000,011,101,110}$ we get $C'={111,100,010,001}$ which is not linear.
    \item \textbf{Yes.} Because XOR is basically sum operation and if we sum two binary linear codes we get another linear code.
  \end{alphaparts}
\end{numedquestion}

\begin{numedquestion}
  \begin{alphaparts}
    \item 
    $G$ is a $k \times n $ matrix \\
    $k = 2$ \\
    $n = 5$ \\
    $d = 2$ \\
    \item 
        \begin{tabular}{ |c|c|c|c| } 
            \hline
            00000 & cell2 & cell3 \\ 
            \hline
            00001 & cell5 & cell6 \\ 
            \hline
            00010 & cell8 & cell9 \\ 
            \hline
            00100 & cell8 & cell9 \\ 
            \hline
            01000 & cell8 & cell9 \\ 
            \hline
            10000 & cell8 & cell9 \\ 
            \hline
            11000 & cell8 & cell9 \\ 
            \hline
            00011 & cell8 & cell9 \\ 
            \hline
        \end{tabular}
    \item 
  \end{alphaparts}
\end{numedquestion}


% if you do not solve some of the questions use this command to increment counter
% \setcounter{questionCounter}{4}
% \begin{numedquestion}
%   Questions 2 and 3 were not solved, this is an answer to question 5.
% \end{numedquestion}


% if questions have subparts, use this command
\begin{numedquestion}
  \begin{alphaparts}
    \item \textbf{not equivalent}, second column of $C_2$ cannot be mapped to any of the columns in $C_1$
    \item \textbf{not equivalent}, first column of $C_2$ cannot be mapped to any of the columns in $C_1$
  \end{alphaparts}
\end{numedquestion}

\begin{numedquestion}
  \begin{proof}
 
  \end{proof}
\end{numedquestion}

\begin{numedquestion}
  \begin{alphaparts}
    \item 19054x
      \begin{enumerate}[label=(\roman*)]
        \item 29058x
        \item $2+9+0+5+8+x = 24 + x$
        \item $(24 + x)$ $mod$ $10 = 0$
        \item $x = 6$
      \end{enumerate}
    \item \textbf{No.} For example receiving 0 istead of 5 on position that is being multiplied by 2 will be unnoticed by the validating computation.
    \item \textbf{Yes.} The validating computation is able to detect error when adjanced digits are being switched. However this error couldn't occure more than once in the received code. Otherwise we might not be able to detect it.
  \end{alphaparts}
\end{numedquestion}

\begin{numedquestion}
\begin{center}
\begin{tabular}[c]{ | l | c | }
  \hline			
  A & 0 \\
  B & 100 \\
  C & 101 \\
  D & 11 \\
  \hline  
\end{tabular}
\end{center}
Average code length $L(C) = 1.8$ \\
Efficiency $S(X) = 1.78$
\end{numedquestion}



\end{document}